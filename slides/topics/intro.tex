{
    \metroset{sectionpage=none}
    \section*{Introduction}
}
\begin{frame}{Introduction}
    \begin{columns}
        \begin{column}{0.5\textwidth}
            \begin{figure}[h]
                \centering
                \includegraphics[width=0.5\textwidth]{images/flutter.pdf}
            \end{figure}
        \end{column}
        \begin{column}{0.5\textwidth}
            \begin{figure}[h]
                \centering
                \includegraphics[width=0.5\textwidth]{images/dart.pdf}
            \end{figure}
        \end{column}
    \end{columns}

    \vspace{4ex}

    \begin{itemize}
        \item Mobile framework for creating native(ish)\footnotemark[1] apps for Mobile (Android, iOS), Desktop (Win, Mac, Linux), Web, and Embedded
        \item Write once (dart), run everywhere(ish)\footnotemark[1]
    \end{itemize}

    \footnotetext[1]{Flutter can access native API calls via native platform code (i.e. Android SDK). This can be somewhat abstracted by using cross-platform dart packages.}
    
\end{frame}

\begin{frame}{Prerequisites}
    \begin{itemize}
        \item Programming basics (OOP familiarity recommended)
        \item Having followed Flutter's installation steps sent by e-mail
        \begin{itemize}
            \item \href{https://docs.flutter.dev/get-started/install}{Install Flutter} (Get the Flutter SDK and Android Setup)
            \item \href{https://docs.flutter.dev/get-started/editor}{Set up an editor} (Recommended: VS Code)
        \end{itemize}
    \end{itemize}
\end{frame}